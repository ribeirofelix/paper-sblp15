%% LyX 2.1.0 created this file.  For more info, see http://www.lyx.org/.
%% Do not edit unless you really know what you are doing.
\documentclass[english]{article}
\usepackage[T1]{fontenc}
\usepackage[latin9]{inputenc}
\usepackage{mathtools}
\usepackage{amsmath}
\usepackage{babel}
\begin{document}
In this section, we specify the behavior of \texttt{lua2ast.toAST()}
and \texttt{lua2ast.compile()} by using the formalization of a subset
of Lua semantics, presented in \cite{DeVito2013Terra} as Lua Core.
We use the same formal framework of that work in order to properly
compare and contrast our approach for multi-stage programming to that
employed by Terra.

Lua Core depicts the notions of lexical scoping, closures and side-effects
present in Lua, and is therefore mostly sufficient for our purposes.
We extend this specification with an arbitrary ``binary operator''
expression, mimicking Lua operators supported by Lua2AST. This way,
we have a recursive rule through which we can model Lua expressions
as trees, to be later converted to ASTs. We also include toAST() and
compile() as core language operations so we can specify their semantics
separately from plain functions.

\begin{figure}[t]
\begin{eqnarray*}
e & = & b\,|\, x\,|\,\mbox{let \ensuremath{x=e}\ in\ \ensuremath{e}}\,|\, x\coloneqq e\,|\, e(e)\,|\,\mbox{fun}(x)\{e\}\,|\, e\mbox{ op }e\,|\,\mbox{toAST}(e)\,|\,\mbox{compile}(a)\\
v & = & b\,|\,\left\langle \Gamma,x,e\right\rangle \,|\, a\\
a & = & [\mbox{fn\ \ensuremath{x}\ \ensuremath{a}}]\,|\,[\mbox{base\ \ensuremath{b}}]\,|\,[\mbox{var}\ x\ \left\langle \Gamma,x,e\right\rangle ]\,|\,[\mbox{op\ \ensuremath{a}\ \ensuremath{a}}]
\end{eqnarray*}
\protect\caption{\label{fig:LuaCoreSyntax}Syntax of our version of Lua Core, extended
with constructs to specify Lua2AST}
\end{figure}


The syntax of our version of Lua Core is presented in Figure \ref{fig:LuaCoreSyntax}.
A Lua expression ($e$) can be either a base value ($b$), a variable
($x$), a scoped variable definition ($\mbox{let \ensuremath{x=e}\ in\ \ensuremath{e}}$,
with $e_{1};e_{2}$ as sugar for $\mbox{let \ensuremath{\_=e_{1}}in \ensuremath{e_{2}}}$),
a variable assignment ($x\coloneqq e$), an application ($e(e)$),
a function definition ($\mbox{fun}(x)\{e\}$), an operation on expressions
($e\mbox{ op }e$), or one of the special invocations \emph{toAST($e$)}
and \emph{compile($a$)}. Lua values ($v$) can be base values ($b$),
Lua ASTs ($a$) or closures. A closure is represented as a triple
$\left\langle \Gamma,x,e\right\rangle $, consisting of a namespace
$\Gamma:x\rightarrow p$ (mapping variable names $x$ to memory positions
$p$), an input argument $x$ and an expression body $e$. A Lua AST
for a function consists of a root node ($[\mbox{fn\ \ensuremath{x}\ \ensuremath{a}}]$)
which may contain nodes that wrap base values\emph{ }($[\mbox{base }b]$\emph{),
}operations ($[\mbox{op}\ a\ a]$), and variables ($[\mbox{var}\ x\ \left\langle \Gamma,x,e\right\rangle ]$).
As we will see below, the fact that variables are wrapped by a node
containing a closure is central to our approach.

\begin{figure}[t]
{\footnotesize{}}%
\begin{minipage}[t]{0.59\columnwidth}%
{\footnotesize{}
\[
v,\Sigma\overset{L}{\rightarrow}v,\Sigma\textsc{ (LVal)}
\]
}{\footnotesize \par}

{\footnotesize{}
\[
\frac{\Sigma=(\Gamma,S)}{x,\Sigma\overset{L}{\rightarrow}S(\Gamma(x)),\Sigma}\textsc{ (LVar)}
\]
}{\footnotesize \par}

{\footnotesize{}
\[
\frac{\begin{array}[t]{c}
e_{1},\Sigma_{1}\overset{L}{\rightarrow}v_{1},(\Gamma_{2},S_{2})\,\,\,\,\,\,\,\, p\,\mbox{fresh}\\
e_{2},(\Gamma_{2}[x\leftarrow p],S_{2}[p\leftarrow v_{1}])\overset{L}{\rightarrow}v_{2},(\Gamma_{3},S_{3})
\end{array}}{\mbox{let \ensuremath{x=e_{1}}\,\ in\,\ensuremath{e_{2}}},\Sigma_{1}\overset{L}{\rightarrow}v_{2},(\Gamma_{2},S_{3})}\textsc{ (LLet)}
\]
}{\footnotesize \par}

{\footnotesize{}
\[
\frac{\begin{array}[t]{c}
e_{1},\Sigma_{1}\overset{L}{\rightarrow}\left\langle \Gamma_{1},x,e_{3}\right\rangle ,\Sigma_{2}\\
e_{2},\Sigma_{2}\overset{L}{\rightarrow}v_{1},(\Gamma_{3},S_{3})\,\,\,\,\,\,\,\,\mbox{\ensuremath{p}\ fresh}\\
e_{3},(\Gamma_{1}[x\leftarrow p],S_{3}[p\leftarrow v_{1}])\overset{L}{\rightarrow}v_{2},(\Gamma_{4},S_{4})
\end{array}}{e_{1}(e_{2}),\Sigma_{1}\overset{L}{\rightarrow}v_{2},(\Gamma_{3},S_{4})}\textsc{ (LApp)}
\]
}{\footnotesize \par}

{\footnotesize{}
\[
\frac{e_{1},\Sigma_{1}\overset{L}{\rightarrow}v,(\Gamma,S)\,\,\,\,\,\,\,\,\Gamma(x)=p}{x\coloneqq e,\Sigma\overset{L}{\rightarrow}v,(\Gamma,S[p\leftarrow v])}\textsc{ (LAsn)}
\]
}{\footnotesize \par}

{\footnotesize{}
\[
\frac{\Sigma=(\Gamma,S)}{\mbox{fun(\ensuremath{x})\{\ensuremath{e}\}},\Sigma\overset{L}{\rightarrow}\left\langle \Gamma,x,e\right\rangle ,\Sigma}\textsc{ (LFun)}
\]
}{\footnotesize \par}

{\footnotesize{}
\[
\frac{\begin{array}[t]{c}
e_{1},\Sigma_{1}\overset{L}{\rightarrow}v_{1},\Sigma_{2}\,\,\,\,\,\,\,\, e_{2},\Sigma_{2}\overset{L}{\rightarrow}v_{2},\Sigma_{3}\\
v_{3}=Op(v_{1},v_{2})
\end{array}}{e_{1}\mbox{\,\ op\,}e_{2},\Sigma_{1}\overset{L}{\rightarrow}v_{3},\Sigma_{3}}\textsc{ (LOp)}
\]
}{\footnotesize \par}

{\footnotesize{}
\[
\frac{e_{1},\Sigma_{1}\overset{L}{\rightarrow}\left\langle \Gamma,x,e_{2}\right\rangle ,\Sigma_{2}\,\,\,\,\,\,\,\,\left\langle \Gamma,x,e_{2}\right\rangle ,\Sigma_{2}\overset{D}{\rightarrow}a}{\mbox{toAST(\ensuremath{e_{1}})},\Sigma_{1}\overset{L}{\rightarrow}a,\Sigma_{2}}\textsc{ (LAst)}
\]
}{\footnotesize \par}

{\footnotesize{}
\[
\frac{a,\Sigma_{1}\overset{C}{\rightarrow}e,\Sigma_{2}\,\,\,\,\, e\overset{L}{\rightarrow}v}{\mbox{compile(\ensuremath{a})},\Sigma_{1}\overset{L}{\rightarrow}v,\Sigma_{2}}\textsc{ (LComp)}
\]
}%
\end{minipage}{\footnotesize{}}%
\begin{minipage}[t]{0.4\columnwidth}%
{\footnotesize{}
\[
b,\Sigma\overset{D}{\rightarrow}[\mbox{base }b]\textsc{ (DBase)}
\]
}{\footnotesize \par}

{\footnotesize{}
\[
\frac{\begin{array}[t]{c}
\Sigma=(\Gamma,S)\end{array}}{x,\Sigma\overset{D}{\rightarrow}[\mbox{var}\ x\ \left\langle \Gamma,\_,x\right\rangle ]}\textsc{ (DVar)}
\]
}{\footnotesize \par}

{\footnotesize{}
\[
\frac{e_{1},\Sigma\overset{D}{\rightarrow}a_{1}\,\,\,\,\,\,\,\, e_{2},\Sigma\overset{D}{\rightarrow}a_{2}}{e_{1}\mbox{ op }e_{2},\Sigma\overset{D}{\rightarrow}[\mbox{op\,\ensuremath{a_{1}\,}\ensuremath{a_{2}}}]}\textsc{ (DOp)}
\]
}{\footnotesize \par}

{\footnotesize{}
\[
\frac{e,\Sigma\overset{D}{\rightarrow}a}{\left\langle \Gamma,x,e\right\rangle ,\Sigma\overset{D}{\rightarrow}[\mbox{fn }x\, a]}\textsc{ (DFn)}
\]
}{\footnotesize \par}

{\footnotesize{}
\[
[\mbox{base \ensuremath{b}}],\Sigma\overset{C}{\rightarrow}b,\Sigma\textsc{ (CBase)}
\]
}{\footnotesize \par}

{\footnotesize{}
\[
\frac{\begin{array}[t]{c}
\Sigma_{1}=(\Gamma,S)\\
p\,\mbox{fresh}\\
\Sigma_{2}=(\Gamma[x\leftarrow p],S[p\leftarrow f])
\end{array}}{[\mbox{var}\ x\ f],\Sigma_{1}\overset{C}{\rightarrow}x(\_),\Sigma_{2}}\textsc{ (CVar)}
\]
}{\footnotesize \par}

{\footnotesize{}
\[
\frac{a_{1},\Sigma_{1}\overset{C}{\rightarrow}e_{1},\Sigma_{2}\,\,\,\,\,\,\,\, a_{2},\Sigma_{2}\overset{C}{\rightarrow}e_{2},\Sigma_{3}}{[\mbox{op\,\ensuremath{a_{1}}\ensuremath{\, a_{2}}}],\Sigma_{1}\overset{C}{\rightarrow}e_{1}\mbox{ op }e_{2},\Sigma_{3}}\textsc{ (COp)}
\]
}{\footnotesize \par}

{\footnotesize{}
\[
\frac{\begin{array}[t]{c}
\Sigma_{1}=(\Gamma_{1},S_{1})\\
a,\Sigma_{1}\overset{C}{\rightarrow}e,(\Gamma_{2},S_{2})
\end{array}}{[\mbox{fn }x\, a],\Sigma_{1}\overset{C}{\rightarrow}\left\langle \Gamma_{2},x,e\right\rangle ,(\Gamma_{1},S_{2})}\textsc{ (CFn)}
\]
}%
\end{minipage}{\footnotesize \par}

\protect\caption{\label{fig:Semantics}Rules $\protect\overset{L}{\rightarrow}$ for
the evaluation of Lua expressions, $\protect\overset{D}{\rightarrow}$
for decompiling Lua expressions into ASTs, and $\protect\overset{C}{\rightarrow}$
for compiling ASTs back into expressions.}
\end{figure}


In Figure \ref{fig:Semantics}, we present the rules for evaluating
Lua Core over an environment $\Sigma$, which is a tuple $(\Gamma,S)$
containing a namespace $\Gamma:x\rightarrow p$ and a store $S:p\rightarrow v$
that maps memory positions to values%
\footnote{The semantics of Lua Core in \cite{DeVito2013Terra} is based on an
environment $\Sigma=(\Gamma,S,F)$ where $F$ is specific to Terra
functions. In our presentation, we removed $F$. Rules reused from
\cite{DeVito2013Terra} were adapted accordingly.%
}. We use $\overset{L}{\rightarrow}:(e\times\Sigma)\rightarrow(v\times\Sigma)$
for the evaluation of Lua expressions as in \cite{DeVito2013Terra}.
Rules for $\overset{L}{\rightarrow}$ presented here are equivalent
to those in that work: \textsc{LVal} and \textsc{LVar} evaluate values
and variables; \textsc{LLet} describes variable scoping, by evaluating
$e_{2}$ in an environment created by adding the result of evaluating
$e_{1}$ and assigning it to local variable $x$; \textsc{LApp} describes
function application, propagating side effects; \textsc{LAsn} evaluates
assignments; \textsc{LFun} evaluates function declarations. Our work
adds new rules for $\overset{L}{\rightarrow}$: \textsc{LOp} describes
the evaluation of an arbitrary binary operator, with semantics given
by some function $Op()$; \textsc{LAst} describes the evaluation of
$toAST()$; \textsc{LComp} evaluates $compile()$.

We also add two other relations: rules for decompiling a Lua function
into an AST ($\overset{D}{\rightarrow}:(e\times\Sigma)\rightarrow a$)
and rules for compiling ASTs back into Lua functions ($\overset{C}{\rightarrow}:(a\times\Sigma)\rightarrow(e\times\Sigma)$).
These are used in \textsc{LAst} and \textsc{LComp}, respectively.

The decompilation function $\overset{D}{\rightarrow}$ takes an expression
and an environment and produces an AST. Since $toAST()$ is a pure
function, $\Sigma$ does not figure in the codomain of $\overset{D}{\rightarrow}$.
Note that $\overset{D}{\rightarrow}$ is defined only for base values
(\textsc{DBase}), variables (\textsc{DVar}), the binary operator (\textsc{DOp}),
and the initial function (\textsc{DFn}), mirroring the implementation
of \emph{toAST}() in Lua2AST, which only supports functions containing
these elements. Its rules deconstruct the body of the function and
build the corresponding AST. Of particular interest is rule \textsc{DVar},
which stores in the AST node a newly created closure, which returns
the value of $x$ given the original function's environment.

The compilation function $\overset{C}{\rightarrow}$ takes an AST
and an environment and produces a closure and a new environment. For
each of the four decompilation rules there is a complementary compilation
rule: \textsc{CBase}, \textsc{CVar}, \textsc{COp} and \textsc{CFn}.
Rule \textsc{CVar} translates nodes representing variable references
into a function call to the closure created by rule \textsc{DVar}.
\textsc{CVar} assigns this closure to a variable $x$ in the resulting
environment, and produces a function call to this closure instead
of a variable reference. Rule \textsc{CFn} returns a closure representing
the entire compiled function and a new environment. This environment
contains an unmodified namespace $\Gamma_{1}$ and a new store $S_{2}$,
which includes any closures created for keeping variable references.
The extended namespace $\Gamma_{2}$ produced by the compilation is
used as the namespace of the resulting function's closure.

As a result of running $compile()$, all variable references that
existed in the original function that was decompiled and was now recompiled
were replaced by calls to newly-created closures that merely return
the value of the corresponding variables. These closures use the original
namespace from decompilation time ($\Gamma$ in \textsc{DVar}), so
the variable references are bound to the addresses they have in the
lexical scope where decompilation takes place. Any variable $x$ stored
in an AST will only be evaluated when the compiled function returned
by $compile(a)$ is called. By replacing variable references to function
calls to the wrapper closures, we ensure that the evaluation of variables
(ultimately happening within the wrapper closures) are based on their
original namespaces.
\begin{thebibliography}{1}
\bibitem{DeVito2013Terra}DeVito et al. ''Terra: a multi-stage language
for high-performance computing'' PLDI'13.\end{thebibliography}

\end{document}
