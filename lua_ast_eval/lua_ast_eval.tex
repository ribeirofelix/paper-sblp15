%% LyX 2.1.0 created this file.  For more info, see http://www.lyx.org/.
%% Do not edit unless you really know what you are doing.
\documentclass[english]{article}
\usepackage[T1]{fontenc}
\usepackage[latin9]{inputenc}
\usepackage{mathtools}
\usepackage{amsmath}
\usepackage{babel}
\begin{document}
In this section, we specify below the behavior of toAST() and compile()
by using the formalization of a subset of Lua semantics, presented
in \cite{DeVito2013Terra} as Lua Core. We use the same formal framework
of that work in order to properly compare and contrast our approach
for multi-stage programming to that employed by Terra.

Lua Core depicts the notions of lexical scoping, closures and side-effects
present in Lua, and is therefore mostly sufficient for our purposes.
We extend this specification with an arbitrary ``binary operator''
expression, mimicking Lua operators supported by Lua2AST. This way,
we have a recursive rule through which we can model Lua expressions
as trees, to be later converted to ASTs. We also include toAST() and
compile() as core language operations so we can specify their semantics
separately from plain functions.

\begin{figure}[t]
\begin{eqnarray*}
e & = & b\,|\, x\,|\,\mbox{let \ensuremath{x=e}\,\ in\,\ensuremath{e}}\,|\, x\coloneqq e\,|\, e(e)\,|\,\mbox{fun}(x)\{e\}\,|\, e\mbox{ op }e\,|\,\mbox{toAST}(e)\,|\,\mbox{compile}(a)\\
v & = & b\,|\,\left\langle \Gamma,x,e\right\rangle \,|\, a\\
a & = & [\mbox{fn\,\ensuremath{x}\,\ensuremath{a}}]\,|\,[\mbox{base \ensuremath{b}}]\,|\,[\mbox{var}\, x\,\left\langle \Gamma,x,e\right\rangle ]\,|\,[\mbox{op\,\ensuremath{a}\,\ensuremath{a}}]
\end{eqnarray*}
\protect\caption{\label{fig:LuaCoreSyntax}Syntax of our version of Lua Core, extended
with constructs to specify Lua2AST}
\end{figure}


The syntax of our version of Lua Core is presented in Figure \ref{fig:LuaCoreSyntax}.
Lua expressions ($e$) can be base values ($b$), variables ($x$),
a scoped variable definition ($\mbox{let \ensuremath{x=e}\,\ in\,\ensuremath{e}}$,
with $e;e$ as sugar for $\mbox{let \ensuremath{\_=e}\,\ in\,\ensuremath{e}}$),
a variable assignment ($x\coloneqq e$), an application ($e(e)$),
a function definition ($\mbox{fun}(x)\{e\}$) or an operation on expressions
($e\mbox{ op }e$, with semantics defined by a function $Op$). We
extend this by adding operations \emph{toAST($e$)} and \emph{compile($a$)}.
Lua values ($v$) can be base values ($b$), closures ($\left\langle \Gamma,x,e\right\rangle $)
or Lua ASTs ($a$). A Lua AST for a function consists of a root node
($[\mbox{fn\,\ensuremath{x}\,\ensuremath{a}}]$) which may contain
nodes that wrap base values\emph{ }($[\mbox{base }b]$\emph{), }variables
($[\mbox{var }x\,\left\langle \Gamma,x,e\right\rangle ]$) and operations
($[\mbox{op}\, a\, a]$).

\begin{figure}[t]
{\footnotesize{}}%
\begin{minipage}[t]{0.59\columnwidth}%
{\footnotesize{}
\[
v,\Sigma\overset{L}{\rightarrow}v,\Sigma\textsc{ (LVal)}
\]
}{\footnotesize \par}

{\footnotesize{}
\[
\frac{\Sigma=(\Gamma,S)}{x,\Sigma\overset{L}{\rightarrow}S(\Gamma(x)),\Sigma}\textsc{ (LVar)}
\]
}{\footnotesize \par}

{\footnotesize{}
\[
\frac{\begin{array}[t]{c}
e_{1},\Sigma_{1}\overset{L}{\rightarrow}v_{1},(\Gamma_{2},S_{2})\,\,\,\,\,\,\,\, p\,\mbox{fresh}\\
e_{2},(\Gamma_{2}[x\leftarrow p],S_{2}[p\leftarrow v_{1}])\overset{L}{\rightarrow}v_{2},(\Gamma_{3},S_{3})
\end{array}}{\mbox{let \ensuremath{x=e_{1}}\,\ in\,\ensuremath{e_{2}}},\Sigma_{1}\overset{L}{\rightarrow}v_{2},(\Gamma_{2},S_{3})}\textsc{ (LLet)}
\]
}{\footnotesize \par}

{\footnotesize{}
\[
\frac{\begin{array}[t]{c}
e_{1},\Sigma_{1}\overset{L}{\rightarrow}\left\langle \Gamma_{1},x,e_{3}\right\rangle ,\Sigma_{2}\\
e_{2},\Sigma_{2}\overset{L}{\rightarrow}v_{1},(\Gamma_{3},S_{3})\,\,\,\,\,\,\,\,\mbox{\ensuremath{p}\ fresh}\\
e_{3},(\Gamma_{1}[x\leftarrow p],S_{3}[p\leftarrow v_{1}])\overset{L}{\rightarrow}v_{2},(\Gamma_{4},S_{4})
\end{array}}{e_{1}(e_{2}),\Sigma_{1}\overset{L}{\rightarrow}v_{2},(\Gamma_{3},S_{4})}\textsc{ (LApp)}
\]
}{\footnotesize \par}

{\footnotesize{}
\[
\frac{e_{1},\Sigma_{1}\overset{L}{\rightarrow}v_{1},(\Gamma,S)\,\,\,\,\,\,\,\,\Gamma(x)=p}{x\coloneqq e,\Sigma\overset{L}{\rightarrow}v,(\Gamma,S[p\leftarrow v])}\textsc{ (LAsn)}
\]
}{\footnotesize \par}

{\footnotesize{}
\[
\frac{\Sigma=(\Gamma,S)}{\mbox{fun(\ensuremath{x})\{\ensuremath{e}\}},\Sigma\overset{L}{\rightarrow}\left\langle \Gamma,x,e\right\rangle ,\Sigma}\textsc{ (LFun)}
\]
}{\footnotesize \par}

{\footnotesize{}
\[
\frac{\begin{array}[t]{c}
e_{1},\Sigma_{1}\overset{L}{\rightarrow}v_{1},\Sigma_{2}\,\,\,\,\,\,\,\, e_{2},\Sigma_{2}\overset{L}{\rightarrow}v_{2},\Sigma_{3}\\
v_{3}=Op(v_{1},v_{2})
\end{array}}{e_{1}\mbox{\, op\,}e_{2},\Sigma_{1}\overset{L}{\rightarrow}v_{3},\Sigma_{3}}\textsc{ (LOp)}
\]
}{\footnotesize \par}

{\footnotesize{}
\[
\frac{e_{1},\Sigma_{1}\overset{L}{\rightarrow}\left\langle \Gamma,x,e_{2}\right\rangle ,\Sigma_{2}\,\,\,\,\,\,\,\,\left\langle \Gamma,x,e_{2}\right\rangle ,\Sigma_{2}\overset{D}{\rightarrow}a}{\mbox{toAST(\ensuremath{e_{1}})},\Sigma_{1}\overset{L}{\rightarrow}a,\Sigma_{2}}\textsc{ (LAst)}
\]
}{\footnotesize \par}

{\footnotesize{}
\[
\frac{a,\Sigma_{1}\overset{C}{\rightarrow}e,\Sigma_{2}}{\mbox{compile(\ensuremath{a})},\Sigma_{1}\rightarrow e,\Sigma_{2}}\textsc{ (LComp)}
\]
}%
\end{minipage}{\footnotesize{}}%
\begin{minipage}[t]{0.4\columnwidth}%
{\footnotesize{}
\[
b,\Sigma\overset{D}{\rightarrow}[\mbox{base }b]\textsc{ (DBase)}
\]
}{\footnotesize \par}

{\footnotesize{}
\[
\frac{\begin{array}[t]{c}
\Sigma=(\Gamma,S)\\
x'\,\mbox{fresh}
\end{array}}{x,\Sigma\overset{D}{\rightarrow}[\mbox{var }x'\,\left\langle \Gamma,\_,x\right\rangle ]}\textsc{ (DVar)}
\]
}{\footnotesize \par}

{\footnotesize{}
\[
\frac{e_{1},\Sigma\overset{D}{\rightarrow}a_{1}\,\,\,\,\,\,\,\, e_{2},\Sigma\overset{D}{\rightarrow}a_{2}}{e_{1}\mbox{ op }e_{2},\Sigma\overset{D}{\rightarrow}[\mbox{op\,\ensuremath{a_{1}\,}\ensuremath{a_{2}}}]}\textsc{ (DOp)}
\]
}{\footnotesize \par}

{\footnotesize{}
\[
\frac{e,\Sigma\overset{D}{\rightarrow}a}{\left\langle \Gamma,x,e\right\rangle ,\Sigma\overset{D}{\rightarrow}[\mbox{fn }x\, a]}\textsc{ (DFn)}
\]
}{\footnotesize \par}

{\footnotesize{}
\[
[\mbox{base \ensuremath{b}}],\Sigma\overset{C}{\rightarrow}b,\Sigma\textsc{ (CBase)}
\]
}{\footnotesize \par}

{\footnotesize{}
\[
\frac{\begin{array}[t]{c}
\Sigma_{1}=(\Gamma,S)\\
p\,\mbox{fresh}\\
\Sigma_{2}=(\Gamma[x'\leftarrow p],S[p\leftarrow f])
\end{array}}{[\mbox{var }x'\, f],\Sigma_{1}\overset{C}{\rightarrow}x'(\_),\Sigma_{2}}\textsc{ (CVar)}
\]
}{\footnotesize \par}

{\footnotesize{}
\[
\frac{a_{1},\Sigma_{1}\overset{C}{\rightarrow}e_{1},\Sigma_{2}\,\,\,\,\,\,\,\, a_{2},\Sigma_{2}\overset{C}{\rightarrow}e_{2},\Sigma_{3}}{[\mbox{op\,\ensuremath{a_{1}}\ensuremath{\, a_{2}}}],\Sigma_{1}\overset{C}{\rightarrow}e_{1}\mbox{ op }e_{2},\Sigma_{3}}\textsc{ (COp)}
\]
}{\footnotesize \par}

{\footnotesize{}
\[
\frac{\begin{array}[t]{c}
\Sigma_{1}=(\Gamma_{1},S_{1})\\
a,\Sigma_{1}\overset{C}{\rightarrow}e,(\Gamma_{2},S_{2})
\end{array}}{[\mbox{fn }x\, a],\Sigma_{1}\overset{C}{\rightarrow}\left\langle \Gamma_{2},x,e\right\rangle ,(\Gamma_{1},S_{2})}\textsc{ (CFn)}
\]
}%
\end{minipage}{\footnotesize \par}

\protect\caption{\label{fig:Semantics}Rules $\protect\overset{L}{\rightarrow}$ for
the evaluation of Lua expressions, $\protect\overset{D}{\rightarrow}$
for decompiling Lua expressions into ASTs, and $\protect\overset{C}{\rightarrow}$
for compiling ASTs back into expressions.}
\end{figure}


In Figure \ref{fig:Semantics}, we present the rules for evaluating
Lua Core over an environment $\Sigma$, which is a tuple $(\Gamma,S)$
containing a namespace $\Gamma:x\rightarrow p$ and a store $S:p\rightarrow v$
(where $p$ are memory positions). We use $\overset{L}{\rightarrow}$
for the evaluation of Lua expressions as in \cite{DeVito2013Terra};
where rules have the same names, they have the same semantics as those
presented in that work. We added rules \textsc{LOp} for the binary
operation, \textsc{LAst} for the $toAST()$ function and \textsc{LComp}
for the \emph{compile}() function.

We also add two other relations: rules for decompiling a Lua function
into an AST ($\overset{D}{\rightarrow}:(e\times\Sigma)\rightarrow a$)
and rules for compiling ASTs back into Lua functions ($\overset{C}{\rightarrow}:(a\times\Sigma)\rightarrow(e\times\Sigma)$).
In our formalization, decompiling the function produces an AST but
does not affect the environment; all values produced are stored in
the AST itself. Compiling returns an environment with an unmodified
namespace and a modified store which includes the required closures. 

Note that $\overset{D}{\rightarrow}$ is defined only for variables
(\textsc{DVar}), base values (\textsc{DBase}), the binary operator
(\textsc{DOp}), and the initial function (\textsc{DFn}), mirroring
the implementation of \emph{toAST}() in LuaToAST. Its rules deconstruct
the body of the function and build the corresponding AST. Of particular
interest is rule \textsc{DVar}, which stores in the AST node a fresh
variable $x'$ (i.e. one guaranteed not to exist in $\Gamma$) and
a newly created closure, which returns the value of $x$ given the
original function's environment. In the complementary rule \textsc{CVar},
this new variable $x'$ is assigned in the resulting environment to
hold the node's closure $f$. Note in rule \textsc{CFn} that the derived
namespace $\Gamma_{2}$ is used only in the resulting closure, whereas
the derived store $S_{2}$ is used in the resulting environment. In
$\Gamma_{2}$, all variable references that existed in the original
function that was decompiled and now recompiled were replaced by calls
to newly-created closures that merely return the value of the corresponding
variables.

These closures use the original namespace from decompilation time
($\Gamma$ in \textsc{DVar}), so the variable references are bound
to the addresses they have in the lexical scope where decompilation
takes place. Any variable $x$ stored in an AST will only be evaluated
when the compiled function returned by $compile(a)$ is called, and
the call to wrapper closure $x'()$ will ensure that $x$ is tied
to its original namespace.
\begin{thebibliography}{1}
\bibitem{DeVito2013Terra}DeVito et al. ''Terra: a multi-stage language
for high-performance computing'' PLDI'13.\end{thebibliography}

\end{document}
